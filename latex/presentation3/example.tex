\documentclass[t, screen, aspectratio=43]{beamer}
\usepackage[T1]{fontenc}
\usepackage[utf8]{inputenc}
\usepackage{epsf}
\usepackage{graphicx}
\usepackage{geometry}
\usepackage{tabularx}
\usepackage[table]{colortbl}
% Use the NTNU-temaet for beamer 
% \usetheme[style=ntnu|simple|vertical|horizontal, 
%     language=bm|nn|en, 
%     smalltitle, 
%     city=all|trondheim|alesund|gjovik]{ntnu2017}
\usetheme[style=helvet,language=en]{ntnu2017}

\usepackage[english]{babel}
\usepackage[style=numeric,backend=biber,natbib=false,sorting=none]{biblatex}

\title[Short title]{Ultra-Low Power PLL for Wake-up Receiver Applications}
\subtitle{Specialization Project Progress - 3rd Week}
\author[C Nielsen]{Cole Nielsen}
\institute[NTNU]{Department of Electronic Systems, NTNU}
\date{13 September 2019 (Week 37)}
%\date{} % To have an empty date

\addbibresource{example.bib} % Add bibliography database

% Set the reference style to numeric.
% See here: http://tex.stackexchange.com/questions/68080/beamer-bibliography-icon
\setbeamertemplate{bibliography item}[text] 

% Set bibliography fonts to a small size.
\renewcommand*{\bibfont}{\footnotesize}

\begin{document}

\begin{frame}
	\titlepage%
\end{frame}

% Alternatively, special title page command to get a different background
% \ntnutitlepage

% #############################################################################
% This week
% #############################################################################

\begin{frame}
	\frametitle{This Week}
	\begin{block}{Timeline Tasks}
		\begin{itemize}
			\footnotesize
			\item \textbf{Primary:} Start-up phase to modeling and simulation of PLL.
			\begin{itemize}
				\footnotesize
				\item Implement each PLL component as class.
				\begin{itemize}
					\item Ref. Clock, TDC, loop filter, DCO, divider.
				\end{itemize} 
				\item Simulate system in discrete time (uniform step size):	
				\begin{itemize}
					\footnotesize
					\item Each iteration, call update method of objects representing PLL components. 
					\item Store node values within PLL to arrays.
					\item For now have to used fixed time step for filtering and approach to generating phase noise (random phase walk in DCO phase signal).
				\end{itemize}
				\item Set up Git repository for code, also for report/presentations.
			\end{itemize} 
		\end{itemize}    
	\end{block}
\end{frame}

% #############################################################################
% Python implementation
% #############################################################################

\begin{frame}
	\frametitle{Implementation in Python}
	\begin{block}{Progress}
		\vspace{-.2em}
		\begin{itemize}
			\footnotesize
			\item Ran into issue with my Python installations (3.6/3.7)
			\begin{itemize}
				\footnotesize
				\item Virtual environments was not set up correct when I installed 3.7, so now I have issues with packages not working between 3.6/3.7
				\item I.E. Can't use Control package in 3.6, can't use plotting package in 3.7 
				\item Wasted too much time trying to fix, currently just using 2.7 
			\end{itemize}
			\footnotesize
			\item Have implemented all blocks for PLL; preliminary phase noise plotting/analysis methods.
			\begin{itemize}
				\footnotesize
				\item Can plot SSB phase noise.
				\item Can fit phase noise model with 1/f dependency on freq. (corresponds to random phase walk component that is dominant in ring oscillators).
				\item Need to include level quantization in loop filter. 
			\end{itemize} 

		\end{itemize}    
	\end{block}
\end{frame}

% slide 2

\begin{frame}
	\frametitle{Implementation in Python}
	\begin{block}{Still To Do}
		\vspace{-.2em}
		\begin{itemize}
			\footnotesize
			\item Implement level-quantization in loop filter.
			\item Automatic analysis of performance (lock detection, residual phase modulation, lock-in/pull-in range).
			\item Automatic optimization (using gradient descent) of PLL parameters?
			\item Z-domain modeling of loop? Develop (by hand) some ideal transfer funtions for loop.

		\end{itemize}    
	\end{block}
\end{frame}

% #############################################################################
% Specification
% #############################################################################

\begin{frame}
	\frametitle{Specification (unchanged)}
	\begin{block}{Preliminary Performance Targets}
		\scriptsize
		\begin{table}[h!]
			\centering
			\def\arraystretch{1.5}		
			\setlength\arrayrulewidth{0.75pt}
			\setlength{\tabcolsep}{1em} % for the horizontal padding
			\begin{tabular}{|l|r|l|l|}
				\hline 
				\rule[-1ex]{0pt}{2.5ex} \cellcolor{gray!40}\textbf{Parameter} & \cellcolor{gray!40}\textbf{Value} & \cellcolor{gray!40}\textbf{Unit }& \cellcolor{gray!40}\textbf{Notes}\\ 
				\hline 
				\rule[-1ex]{0pt}{2.5ex} \textbf{Frequency}  & 2.4-2.4835 & GHz & 2.4G ISM Band\\ 
				\hline 
				\rule[-1ex]{0pt}{2.5ex} \textbf{Ref. frequency} & 16 & MHz & Yields 6 channels \\ 
				\hline 
				\rule[-1ex]{0pt}{2.5ex} \textbf{Power} & $\leq$ 100  &$\mu$W & \\ 
				\hline 
				\rule[-1ex]{0pt}{2.5ex} \textbf{Residual FM} & $\leq$ 107  &kHz$_{RMS}$ & BER $\leq$ 1e-2, $f_{dev}$=$\pm$250 KHz\\ 
				\hline 
				\rule[-1ex]{0pt}{2.5ex} \textbf{Initial Lock Time} & $\leq$ 50 & $\mu$s & Upon cold start \\ 
				\hline 
				\rule[-1ex]{0pt}{2.5ex} \textbf{Re-lock Time} & $\leq$ 5 & $\mu$s & Coming out of standby \\ 
				\hline 
				\rule[-1ex]{0pt}{2.5ex} \textbf{Bandwidth} & 100 & kHz & (nominally), tunable \\ 
				\hline 
			\end{tabular} 
			% \caption{Assigned specifications for branch line hybrid design.}
			% \label{asgn_specs}
		\end{table}   
		Additionally: PLL output should support IQ sampling at LO frequency.
	\end{block}    
\end{frame}

% #############################################################################
% Architecture - block diagram
% #############################################################################

\begin{frame}
	\frametitle{Architecture (unchanged)}
	\begin{block}{Block Diagram}
	\center\includegraphics[width=0.75\textwidth, angle=0]{pll2.pdf}

	\end{block}
		\begin{block}{Power Targets}
		\vspace{-.1em}
		\begin{table}[htb!]
			\tiny
			\centering
			\def\arraystretch{1.5}		
			\setlength\arrayrulewidth{0.75pt}
			\setlength{\tabcolsep}{1em} % for the horizontal padding
			\begin{tabular}{|l|l|l|l|l|}
				\hline 
				\rule[-1ex]{0pt}{2.5ex} \cellcolor{gray!40}\textbf{DCO} & \cellcolor{gray!40}\textbf{TDC} & \cellcolor{gray!40}\textbf{Divider }& \cellcolor{gray!40}\textbf{Other} & \cellcolor{gray!40}\textbf{SUM} \\ 
				\hline 
				\rule[-1ex]{0pt}{2.5ex} 70 $\mu$W& 20 $\mu$W & 10 $\mu$W & $<<$ 1 $\mu$W & 100 $\mu$W\\ 
				\hline 
			\end{tabular} 
			% \caption{Assigned specifications for branch line hybrid design.}
			% \label{asgn_specs}
		\end{table}   
	\end{block}

\end{frame}

% #############################################################################
% Timeline
% #############################################################################
\begin{frame}
	\frametitle{Autumn Timeline}
	\begin{table}[htb!]
		\tiny
		\centering
		\def\arraystretch{1.5}		
		\setlength\arrayrulewidth{0.75pt}
		\setlength{\tabcolsep}{1em} % for the horizontal padding
		\begin{tabular}{|l|l|l|l|}
			\hline 
			\rule[-1ex]{0pt}{2.5ex} \cellcolor{gray!40}\textbf{Week Number} & \cellcolor{gray!40}\textbf{Dates} &\cellcolor{gray!40}\textbf{Tasks} & \cellcolor{gray!40}\textbf{Outcomes}\\ 
			\hline 
			\rule[-1ex]{0pt}{2.5ex} \cellcolor{red!20}\textbf{36}& \cellcolor{red!20}2.9 - 8.9 & \cellcolor{red!20}Review PLL Design & \cellcolor{red!20}Refreshed Knowledge\\ 
			\hline 
			\rule[-1ex]{0pt}{2.5ex} \cellcolor{red!20}\textbf{37}& \cellcolor{red!20}9.9 - 15.9 & \cellcolor{red!20}Modeling/simulation (set up) & \cellcolor{red!20}--\\ 
			\hline 
			\rule[-1ex]{0pt}{2.5ex} \textbf{38}& 16.9 - 22.9 & Modeling/simulation & TDC/DCO Requirements\\ 
			\hline 
			\rule[-1ex]{0pt}{2.5ex} \textbf{39}& 23.9 - 29.9& Modeling/simulation& Loop Filter/Digital Algorithms\\ 
			\hline 
			\rule[-1ex]{0pt}{2.5ex} \textbf{40}& 30.9 - 6.10& Modeling/simulation& Ideal (ahdlLib?) implementation in Cadence of PLL\\ 
			\hline 
			\rule[-1ex]{0pt}{2.5ex} \textbf{41}& 7.10 - 13.10& Circuit Research & DCO/Divider topologies\\ 
			\hline 
			\rule[-1ex]{0pt}{2.5ex} \textbf{42}& 14.10 - 20.10& Circuit Research & TDC/other topologies\\ 
			\hline 
			\rule[-1ex]{0pt}{2.5ex} \textbf{43}& 21.10 - 27.10& Circuit Implementation& Digital logic (schematic)\\ 
			\hline 
			\rule[-1ex]{0pt}{2.5ex} \textbf{44}& 28.10 - 3.11& Circuit Implementation& DCO (schematic)\\ 
			\hline 
			\rule[-1ex]{0pt}{2.5ex} \textbf{45}& 4.11 - 10.11& Circuit Implementation& Divider/other (schematic)\\ 
			\hline 
			\rule[-1ex]{0pt}{2.5ex} \textbf{46}& 11.11 - 17.11& Circuit Implementation (TDC)& \\ 
			\hline 
			\rule[-1ex]{0pt}{2.5ex} \textbf{47}& 18.11 - 24.11& Circuit Implementation (TDC)& TDC (schematic)\\ 
			\hline 
			\rule[-1ex]{0pt}{2.5ex} \textbf{48}& 25.11 - 1.12& Full Circuit testing & Testbenches, find bugs, design fixes\\ 
			\hline 
			\rule[-1ex]{0pt}{2.5ex} \textbf{49}& 2.12 - 8.12& Full Circuit testing& Design Fixes/iteration\\ 
			\hline 
			\rule[-1ex]{0pt}{2.5ex} \textbf{50}& 9.12 - 15.12& --& --\\ 
			\hline 
		\end{tabular}
		\begin{flushleft}*I will write the report simultaneously with the work.
		\end{flushleft}
		% \caption{Assigned specifications for branch line hybrid design.}
		% \label{asgn_specs}
	\end{table}   
\end{frame}

% #############################################################################
% project phases
% #############################################################################


\begin{frame}
	\frametitle{Project Phases}
	\begin{block}{Autumn 2019}
		\footnotesize
		\begin{itemize}
			\item System modeling and simulation.
			\begin{itemize}
				\footnotesize
				\item Learn PLL theory in detail
				\item Evaluate feasability of PLL architectures (counter, TDC-based)
				\item Determine requirements for TDC/DCO/Divider/logic (bits of resolution, accuracy etc) to meet PLL performance specifications.
				\item Determine digital logic for loop filter, validate stability and lock time performance.
			\end{itemize}
			\item Research ultra-low power circuit topologies to implement system components that will meet determined requirements.
			\item Translate component-level specifications into schematic-level circuit designs.
			\begin{itemize}
				\footnotesize
				\item Try, fail, try again until functional at schematic level.
				\begin{itemize}
					\footnotesize
					\item I expect the TDC to be difficult.
				\end{itemize}
			\end{itemize}      
		\end{itemize}
	\end{block}
\end{frame}

% #############################################################################
% Project phases slide 2
% #############################################################################


\begin{frame}
	\frametitle{Project Phases (continued)}
	\begin{block}{Spring 2020}
		\begin{itemize}
			\footnotesize
			\item Finalize schematic-level design.
			\item Estabilish thorough tests for PLL performance (automated?) to help in layout.
			\item Layout of PLL.
			\begin{itemize}
				\footnotesize
				\item Design iteration until design specs met.
				\item Probably very time consuming.
			\end{itemize}
			\item Full characterization/validation of design performance. 
			\begin{itemize}
				\footnotesize
				\item Comprehensive Corners/Monte-Carlo testing (time consuming??)
				\item More design iteration if new issues crop up...
			\end{itemize}
			\item Thesis paper writing.
		\end{itemize}
	\end{block}
\end{frame}



\end{document}
